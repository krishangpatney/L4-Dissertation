\chapter{Introduction}

% reset page numbering. Don't remove this!
\pagenumbering{arabic} 


Why should the reader care about what are you doing and what are you actually doing?
\section{Guidance}

\textbf{Motivate} first, then state the general problem clearly. 

\section{Writing guidance}
\subsection{Who is the reader?}

This is the key question for any writing. Your reader:

\begin{itemize}
    \item
    is a trained computer scientist: \emph{don't explain basics}.
    \item
    has limited time: \emph{keep on topic}.
    \item
    has no idea why anyone would want to do this: \emph{motivate clearly}
    \item
    might not know \emph{anything} about your project in particular:
    \emph{explain your project}.
    \item
    but might know precise details and check them: \emph{be precise and
    strive for accuracy.}
    \item
    doesn't know or care about you: \emph{personal discussions are
    irrelevant}.
\end{itemize}

Remember, you will be marked by your supervisor and one or more members
of staff. You might also have your project read by a prize-awarding
committee or possibly a future employer. Bear that in mind.

\subsection{References and style guides}
There are many style guides on good English writing. You don't need to
read these, but they will improve how you write.

\begin{itemize}
    \item
    \emph{How to write a great research paper} \cite{Pey17} (\textbf{recommended}, even though you aren't writing a research paper)
    \item
    \emph{How to Write with Style} \cite{Von80}. Short and easy to read. Available online.
    \item
    \emph{Style: The Basics of Clarity and Grace} \cite{Wil09} A very popular modern English style guide.
    \item
    \emph{Politics and the English Language} \cite{Orw68}  A famous essay on effective, clear writing in English.
    \item
    \emph{The Elements of Style} \cite{StrWhi07} Outdated, and American, but a classic.
    \item
    \emph{The Sense of Style} \cite{Pin15} Excellent, though quite in-depth.
\end{itemize}

\subsubsection{Citation styles}

\begin{itemize}
\item If you are referring to a reference as a noun, then cite it as: ``\citet{Orw68} discusses the role of language in political thought.''
\item If you are referring implicitly to references, use: ``There are many good books on writing \citep{Orw68, Wil09, Pin15}.''
\end{itemize}

% There is a complete guide on good citation practice by Peter Coxhead available here: \url{http://www.cs.bham.ac.uk/~pxc/refs/index.html}. 
If you are unsure about how to cite online sources, please see this guide: \url{https://student.unsw.edu.au/how-do-i-cite-electronic-sources}.

\subsection{Plagiarism warning}

\begin{highlight_title}{WARNING}
    
    If you include material from other sources without full and correct attribution, you are commiting plagiarism. The penalties for plagiarism are severe.
    Quote any included text and cite it correctly. Cite all images, figures, etc. clearly in the caption of the figure.
\end{highlight_title}


%==================================================================================================================================