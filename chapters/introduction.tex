\chapter{Introduction}
% reset page numbering. Don't remove this!
\pagenumbering{arabic} 



    % What will you be investigating (in plain-language, big picture-level)?
    
    % Measuring which deployment pattern works the best for what type of application. 
    % aka Optimising deployment for microservices within the cloud. 
    
    % Why is that worth investigating? How is it important to academia or business? How is it sufficiently original?
    % Microservices are often used by industry (IBM has a stat), this means that a lot of it is deployed on the cloud using customized standards, however, there seems to be a gap of literature that goes over a few patterns and analysis them. 
    % What are your research aims and research question(s)? Note that the research questions can sometimes be presented at the end of the literature review (next chapter).
    % Research aims have been talked about under the objectives sections
    % What is the scope of your study? In other words, what will you cover and what won’t you cover?
    % This has also been covered 
    % How will you approach your research? In other words, what methodology will you adopt?
    % To conduct the research scripts have been used to automate the process 
    % How will you structure your dissertation? What are the core chapters and what will you do in each of them?

% Links
% https://aws.amazon.com/microservices
% https://dl.acm.org/doi/pdf/10.1145/3382025.3414942
% https://microservices.io/patterns/microservices.html
% (https://www.ibm.com/uk-en/cloud/learn/microservices) 

Microservices are an architectural and organizational shift within the software engineering industry, where software built with the architecture comprises of small independent services, which are highly maintainable, testable, loosely coupled and independently deployable. According to a recent survey conducted by IBM, 87\% of developers and IT executives across the industry agree that adopting the microservice architecture is worth their time and effort, as services can be developed by smaller independent teams within an organisation allowing them to freely integrate functionalities from various complex business components to create the specialized services using different technologies to other systems, such as using new programming languages, databases or communication protocols. 

The microservice architecture additionally also appeals to developers due to the several advantages over other architectures such as monoliths, as this offer easier deployment cycles by allowing to integrate with different automation techniques such as continuous integration and continuous delivery (CI/CD), allowing developers to trial new ideas and easily roll back upon failure, making it cost-effective to experiment, update code and deploy to the end-user.


\section{Motivation}
Due to the fact that deployment of services to the cloud can is often automated, the industry has created a multitude of patterns, which allow for microserivce applications to be optimally deployed. The following dissertation aim's to introduce a suitable architecture which can deploy,test and analyse performance of different applications being deployed under different patterns to the cloud. Seeking to answer possible effects the patterns may have on an application's performance, and how the applications react in a scaled up environment. 

\section{Outline}
The following chapters in the dissertation follow the following structure : 

\begin{itemize}
    \item \autoref{chap:background} explores work done by industry and academia to explain key details which are the basis for rest of the dissertation. 
    
    \item \autoref{chap:anr} presents the requirement's and  

\end{itemize}