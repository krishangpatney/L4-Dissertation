
\chapter{Analysis and Requirements}\label{chp:AnR}
This chapter explores the different steps that have been taken to create the following architectures talked in chapter 4 Design, using the MoSCoW technique helped manage requirements throughout the project. Which is followed by a list of objects which have been identified for the dissertation. 
(ONE OF THIS MUST GO!)
\section{Requirements - MoSCoW}
\begin{itemize}
  \item Must
  \begin{itemize}
        \item There must be more than one application to deploy and test.
        \item There must be a load generator that simulates real-user usage. 
        \item There must be a workflow for deploying and testing applications
  \end{itemize}
  \item Should
  \begin{itemize}
    \item There should be at least 3 deployment patterns 
    \item There should be the ability to deploy the applications multiple times in one go. 
    \item There should be the ability to collect all possible metrics for both the machines and API.
  \end{itemize}
    \item Could
  \begin{itemize}
    \item There could be 
  \end{itemize}
\end{itemize}



\section{Research Objectives}
To create a suitable deployment and testing environment for the various patterns, the project is split into the following objectives:  
\begin{itemize}
    \item \textbf{O1} Identify suitable patterns for microservice deployments. 
    \item \textbf{O2} Identify open-source applications based on the microservice architecture to be used for testing patterns. 
    \item \textbf{O3} Create a modular and reproducible setup for deploying applications within different patterns. 
    \item \textbf{O4} Create a modular and reproducible setup for deploying load machines for the applications.
    \item \textbf{O5} Create a workflow for collecting and analysing machine and api metrics.
\end{itemize}